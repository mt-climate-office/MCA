% Options for packages loaded elsewhere
\PassOptionsToPackage{unicode}{hyperref}
\PassOptionsToPackage{hyphens}{url}
\PassOptionsToPackage{dvipsnames,svgnames,x11names}{xcolor}
%
\documentclass[
  letterpaper,
  DIV=11,
  numbers=noendperiod]{scrreprt}

\usepackage{amsmath,amssymb}
\usepackage{lmodern}
\usepackage{iftex}
\ifPDFTeX
  \usepackage[T1]{fontenc}
  \usepackage[utf8]{inputenc}
  \usepackage{textcomp} % provide euro and other symbols
\else % if luatex or xetex
  \usepackage{unicode-math}
  \defaultfontfeatures{Scale=MatchLowercase}
  \defaultfontfeatures[\rmfamily]{Ligatures=TeX,Scale=1}
\fi
% Use upquote if available, for straight quotes in verbatim environments
\IfFileExists{upquote.sty}{\usepackage{upquote}}{}
\IfFileExists{microtype.sty}{% use microtype if available
  \usepackage[]{microtype}
  \UseMicrotypeSet[protrusion]{basicmath} % disable protrusion for tt fonts
}{}
\makeatletter
\@ifundefined{KOMAClassName}{% if non-KOMA class
  \IfFileExists{parskip.sty}{%
    \usepackage{parskip}
  }{% else
    \setlength{\parindent}{0pt}
    \setlength{\parskip}{6pt plus 2pt minus 1pt}}
}{% if KOMA class
  \KOMAoptions{parskip=half}}
\makeatother
\usepackage{xcolor}
\setlength{\emergencystretch}{3em} % prevent overfull lines
\setcounter{secnumdepth}{5}
% Make \paragraph and \subparagraph free-standing
\ifx\paragraph\undefined\else
  \let\oldparagraph\paragraph
  \renewcommand{\paragraph}[1]{\oldparagraph{#1}\mbox{}}
\fi
\ifx\subparagraph\undefined\else
  \let\oldsubparagraph\subparagraph
  \renewcommand{\subparagraph}[1]{\oldsubparagraph{#1}\mbox{}}
\fi


\providecommand{\tightlist}{%
  \setlength{\itemsep}{0pt}\setlength{\parskip}{0pt}}\usepackage{longtable,booktabs,array}
\usepackage{calc} % for calculating minipage widths
% Correct order of tables after \paragraph or \subparagraph
\usepackage{etoolbox}
\makeatletter
\patchcmd\longtable{\par}{\if@noskipsec\mbox{}\fi\par}{}{}
\makeatother
% Allow footnotes in longtable head/foot
\IfFileExists{footnotehyper.sty}{\usepackage{footnotehyper}}{\usepackage{footnote}}
\makesavenoteenv{longtable}
\usepackage{graphicx}
\makeatletter
\def\maxwidth{\ifdim\Gin@nat@width>\linewidth\linewidth\else\Gin@nat@width\fi}
\def\maxheight{\ifdim\Gin@nat@height>\textheight\textheight\else\Gin@nat@height\fi}
\makeatother
% Scale images if necessary, so that they will not overflow the page
% margins by default, and it is still possible to overwrite the defaults
% using explicit options in \includegraphics[width, height, ...]{}
\setkeys{Gin}{width=\maxwidth,height=\maxheight,keepaspectratio}
% Set default figure placement to htbp
\makeatletter
\def\fps@figure{htbp}
\makeatother
\newlength{\cslhangindent}
\setlength{\cslhangindent}{1.5em}
\newlength{\csllabelwidth}
\setlength{\csllabelwidth}{3em}
\newlength{\cslentryspacingunit} % times entry-spacing
\setlength{\cslentryspacingunit}{\parskip}
\newenvironment{CSLReferences}[2] % #1 hanging-ident, #2 entry spacing
 {% don't indent paragraphs
  \setlength{\parindent}{0pt}
  % turn on hanging indent if param 1 is 1
  \ifodd #1
  \let\oldpar\par
  \def\par{\hangindent=\cslhangindent\oldpar}
  \fi
  % set entry spacing
  \setlength{\parskip}{#2\cslentryspacingunit}
 }%
 {}
\usepackage{calc}
\newcommand{\CSLBlock}[1]{#1\hfill\break}
\newcommand{\CSLLeftMargin}[1]{\parbox[t]{\csllabelwidth}{#1}}
\newcommand{\CSLRightInline}[1]{\parbox[t]{\linewidth - \csllabelwidth}{#1}\break}
\newcommand{\CSLIndent}[1]{\hspace{\cslhangindent}#1}

<script type="application/json" class="js-hypothesis-config">
{
  "showHighlights": true
}
</script>
<script async src="https://hypothes.is/embed.js"></script>
\KOMAoption{captions}{tableheading}
\makeatletter
\@ifpackageloaded{tcolorbox}{}{\usepackage[many]{tcolorbox}}
\@ifpackageloaded{fontawesome5}{}{\usepackage{fontawesome5}}
\definecolor{quarto-callout-color}{HTML}{909090}
\definecolor{quarto-callout-note-color}{HTML}{0758E5}
\definecolor{quarto-callout-important-color}{HTML}{CC1914}
\definecolor{quarto-callout-warning-color}{HTML}{EB9113}
\definecolor{quarto-callout-tip-color}{HTML}{00A047}
\definecolor{quarto-callout-caution-color}{HTML}{FC5300}
\definecolor{quarto-callout-color-frame}{HTML}{acacac}
\definecolor{quarto-callout-note-color-frame}{HTML}{4582ec}
\definecolor{quarto-callout-important-color-frame}{HTML}{d9534f}
\definecolor{quarto-callout-warning-color-frame}{HTML}{f0ad4e}
\definecolor{quarto-callout-tip-color-frame}{HTML}{02b875}
\definecolor{quarto-callout-caution-color-frame}{HTML}{fd7e14}
\makeatother
\makeatletter
\makeatother
\makeatletter
\@ifpackageloaded{bookmark}{}{\usepackage{bookmark}}
\makeatother
\makeatletter
\@ifpackageloaded{caption}{}{\usepackage{caption}}
\AtBeginDocument{%
\ifdefined\contentsname
  \renewcommand*\contentsname{Table of contents}
\else
  \newcommand\contentsname{Table of contents}
\fi
\ifdefined\listfigurename
  \renewcommand*\listfigurename{List of Figures}
\else
  \newcommand\listfigurename{List of Figures}
\fi
\ifdefined\listtablename
  \renewcommand*\listtablename{List of Tables}
\else
  \newcommand\listtablename{List of Tables}
\fi
\ifdefined\figurename
  \renewcommand*\figurename{Figure}
\else
  \newcommand\figurename{Figure}
\fi
\ifdefined\tablename
  \renewcommand*\tablename{Table}
\else
  \newcommand\tablename{Table}
\fi
}
\@ifpackageloaded{float}{}{\usepackage{float}}
\floatstyle{ruled}
\@ifundefined{c@chapter}{\newfloat{codelisting}{h}{lop}}{\newfloat{codelisting}{h}{lop}[chapter]}
\floatname{codelisting}{Listing}
\newcommand*\listoflistings{\listof{codelisting}{List of Listings}}
\makeatother
\makeatletter
\@ifpackageloaded{caption}{}{\usepackage{caption}}
\@ifpackageloaded{subcaption}{}{\usepackage{subcaption}}
\makeatother
\makeatletter
\@ifpackageloaded{tcolorbox}{}{\usepackage[many]{tcolorbox}}
\makeatother
\makeatletter
\@ifundefined{shadecolor}{\definecolor{shadecolor}{rgb}{.97, .97, .97}}
\makeatother
\makeatletter
\makeatother
\ifLuaTeX
  \usepackage{selnolig}  % disable illegal ligatures
\fi
\IfFileExists{bookmark.sty}{\usepackage{bookmark}}{\usepackage{hyperref}}
\IfFileExists{xurl.sty}{\usepackage{xurl}}{} % add URL line breaks if available
\urlstyle{same} % disable monospaced font for URLs
\hypersetup{
  pdftitle={DRAFT Climate Change in Montana},
  pdfauthor={Colin Brust},
  colorlinks=true,
  linkcolor={blue},
  filecolor={Maroon},
  citecolor={Blue},
  urlcolor={Blue},
  pdfcreator={LaTeX via pandoc}}

\title{\textbf{DRAFT} Climate Change in Montana}
\author{Colin Brust}
\date{11/2/22}

\begin{document}
\maketitle
\ifdefined\Shaded\renewenvironment{Shaded}{\begin{tcolorbox}[borderline west={3pt}{0pt}{shadecolor}, frame hidden, interior hidden, breakable, boxrule=0pt, sharp corners, enhanced]}{\end{tcolorbox}}\fi

\renewcommand*\contentsname{Table of contents}
{
\hypersetup{linkcolor=}
\setcounter{tocdepth}{2}
\tableofcontents
}
\bookmarksetup{startatroot}

\hypertarget{draft-overview}{%
\chapter*{\texorpdfstring{\textbf{Draft}
Overview}{Draft Overview}}\label{draft-overview}}
\addcontentsline{toc}{chapter}{\textbf{Draft} Overview}

\markboth{\textbf{Draft} Overview}{\textbf{Draft} Overview}

\textbf{Throughout this document, numbers that are in bold still need to
be updated. }

\hypertarget{an-update-to-the-montana-climate-analysis}{%
\section*{An Update to the Montana Climate
Analysis}\label{an-update-to-the-montana-climate-analysis}}
\addcontentsline{toc}{section}{An Update to the Montana Climate
Analysis}

\markright{An Update to the Montana Climate Analysis}

This website is an update to to the climate chapter of the
\href{https://montanaclimate.org/chapter/climate-change}{Montana Climate
Assessment} being developed by the Montana Climate Office. The update to
the assessment provides a summary of historical climate conditions
across the state (Chapter~\ref{sec-historical}), as well as an overview
of projected changes in Montana's climate given different climate change
scenarios (Chapter~\ref{sec-future}). To assess historical climate
conditions, we are using
\href{https://www.ncei.noaa.gov/access/metadata/landing-page/bin/iso?id=gov.noaa.ncdc:C00332}{NClimGrid}
climate data instead of NOAA's
\href{https://www.ncei.noaa.gov/products/land-based-station/us-climate-normals}{station-based
climate normals} used in the original version. We choose to use
NClimGrid because it is spatially continuous across Montana and we have
found it to have high accuracy relative to
\href{https://climate.umt.edu/mesonet/}{Montana Mesonet} weather
stations. Using NClimGrid allows us to provide an accurate and
high-resolution assessment of Montana's historical climate conditions.

To summarize future climate projections, we use data from the sixth
version of the Coupled Model Intercomparison Project (CMIP6), whereas
the original assessment used CMIP5 projections. Specifically, we use an
ensemble of eight models that have been spatially downscaled by the
\href{https://www.nccs.nasa.gov/services/data-collections/land-based-products/nex-gddp-cmip6}{NASA
NEX-GDDP} project. The ensemble of eight models has been shown to best
represent North America's future climate downscaled project ensemble of
8 models that perform well over North America
\href{https://eartharxiv.org/repository/view/2510/}{(https://eartharxiv.org/repository/view/2510/)}.
While CMIP6 projections are similar to the CMIP5 projections used in the
original climate assessment, there are two notable differences:

\begin{enumerate}
\def\labelenumi{\arabic{enumi}.}
\tightlist
\item
  \textbf{Different Downscaling Methods.} Both the original climate
  assessment and the update presented here use downscaled CMIP
  projections. The standard outputs of CMIP models are very coarse in
  spatial resolution, making it difficult to analyze their output at the
  county or even state scale. To overcome this issue, statistical
  methods can be applied to the coarse data to spatially downscale the
  models to a finer spatial resolution. The original climate assessment
  used projections downscaled with the
  \href{https://www.climatologylab.org/maca.html}{Multivariate Adaptive
  Constructed Analogs (MACA)} method, whereas this update uses
  projections downscaled with the
  \href{https://www.nccs.nasa.gov/sites/default/files/NEX-GDDP-CMIP6-Tech_Note.pdf}{Bias-Correction
  Spatial Disaggregation (BCSD)} method. While the results in both the
  original climate assessment and our update are similar, it is
  important to note that these two different downscaling methods were
  used.
\item
  \textbf{Different Climate Scenarios.} In CMIP5 climate projections,
  future climate scenarios were provided as Representative Concentration
  Pathways (RCPs). In short, RCPs simulated the amount of warming that
  would occur given different increases in global CO2 concentrations.
  The CMIP6 projections use a similar grouping of scenarios called
  Shared Socioeconomic Pathways (SSPs; explained in detail in
  Chapter~\ref{sec-future}). SSPs take a more holistic approach to
  future emissions scenarios and incorporate information on how changes
  in the global economy, politics, and population are likely to affect
  CO2 emissions. In this analysis, we use the SSP1-2.6, SSP2-4.5,
  SSP3-7.0 and SSP5-8.5 scenarios to summarize Montana's future climate
  projections. These scenarios encompass the full range of climate
  change scenarios we might expect by the end of the century.
\item
  \textbf{Different Reference Period.} The original climate assessment
  compared future climate conditions to a reference period of 1981 to
  2010. In climate science, it is common practice to update reference
  periods used for comparison with future projections each decade. As
  such, we have updated the reference period used in this analysis to
  the years 1991 to 2020.
\end{enumerate}

Like the original climate assessment, this assessment also uses
Montana's climate divisions to summarize future climate conditions, with
plans to summarize results at the county and watershed scale in the
future.

\hypertarget{importance}{%
\section*{Importance}\label{importance}}
\addcontentsline{toc}{section}{Importance}

\markright{Importance}

Understanding current climate change and projecting future climate
trends are of vital importance--both for our economy and our well-being.
It is our goal to provide science-based information that serves as a
resource for residents of Montana who are interested in understanding
Montana's climate and its impacts on water, agricultural lands and
forests. To provide this understanding, we can learn from past climate
trends. However, knowledge of the past is only partially sufficient in
preparing for a future defined by unprecedented levels of greenhouse
gases in the atmosphere. Therefore, we also provide projections of
change into the future using today's best scientific information and
modeling techniques.

\hypertarget{key-messages}{%
\section*{Key Messages}\label{key-messages}}
\addcontentsline{toc}{section}{Key Messages}

\markright{Key Messages}

Annual average temperatures, including daily minimums, maximums, and
averages, have risen across the state between 1950 and 2015. The average
temperature increase between 1951 and 2015 across Montana was 1.92°F

Winter and spring in Montana have experienced the most warming. Average
temperatures during these seasons have risen by 3.4°F between 1950 and
2015.

Montana's growing season length is increasing due to the earlier onset
of spring and more extended summers; we are also experiencing more warm
days and fewer cool nights. From \textbf{1951-2010}, the growing season
increased by \textbf{12} days. In addition, the annual number of warm
days has increased by \textbf{2.0\%} and the annual number of cool
nights has decreased by \textbf{4.6\%} over this period.

Despite no historical changes in average annual precipitation between
1950 and 2015, there have been changes in average seasonal precipitation
over the same period. Average winter precipitation has decreased by 0.96
inches, which can mostly be attributed to natural variability and an
increase in El Niño events, especially in the western and central parts
of the state. A significant increase in spring precipitation (1.34-1.98
inches) has also occurred during this period for the eastern portion of
the state.

The state of Montana is projected to continue to warm in all geographic
locations, seasons, and under all emission scenarios throughout the 21st
century. By mid century, Montana temperatures are projected to increase
by approximately 3.6-5.5°F depending on the emission scenario. By the
end-of-century, Montana temperatures are projected to increase 3.7-9.2°F
depending on the emission scenario.

The number of days in a year when daily temperature exceeds 90°F (32°C)
and the number of frost-free days are expected to increase across the
state and in both emission scenarios studied. Increases in the number of
days above 90°F (32°C) are expected to be greatest in the eastern part
of the state. Increases in the number of frost-free days are expected to
be greatest in the western part of the state.

Across the state, precipitation is projected to increase in winter,
spring, and fall; precipitation is projected to decrease in summer. The
largest increases are expected to occur during spring in the southern
part of the state. The largest decreases are expected to occur during
summer in the central and eastern parts of the state.

\hypertarget{climate-change-defined}{%
\section*{Climate Change Defined}\label{climate-change-defined}}
\addcontentsline{toc}{section}{Climate Change Defined}

\markright{Climate Change Defined}

The US Global Change Research Program (\textsuperscript{1}) defines
climate change as follows:

\begin{quote}
``Changes in average weather conditions that persist over multiple
decades or longer. Climate change encompasses both increases and
decreases in temperature, as well as shifts in precipitation, changing
risk of certain types of severe weather events, and changes to other
features of the climate system.''
\end{quote}

\hypertarget{outline}{%
\section*{Outline}\label{outline}}
\addcontentsline{toc}{section}{Outline}

\markright{Outline}

This document focuses on three areas:

\begin{enumerate}
\def\labelenumi{\arabic{enumi}.}
\item
  providing a baseline summary of climate and climate change for
  Montana---with a focus on changes in temperature, precipitation, and
  extreme events---including reviewing the fundamentals of climate
  change science;
\item
  reviewing historical trends in Montana's climate, and what those
  trends reveal about how our climate has changed in the past century,
  changes that are potentially attributable to world-wide increases in
  greenhouse gases; and
\item
  considering what today's best available climate models project
  regarding Montana's future, and how certain we can be in those
  projections.
\end{enumerate}

This chapter serves as a foundation for the Montana Climate Assessment,
providing information on present-day climate and climate terminology,
past climate trends, and future climate projections. This foundation
then serves as the basis for analyzing three key sectors of
Montana---water, forests, and agriculture---considered in the other
chapters of this assessment. In the sections below, we introduce the
climate science and discuss important fundamental processes that
determine whether climate remains constant or changes.

\bookmarksetup{startatroot}

\hypertarget{sec-causes}{%
\chapter{\texorpdfstring{\textbf{Draft} Natural and Human Causes of
Climate
Change}{Draft Natural and Human Causes of Climate Change}}\label{sec-causes}}

Climate is driven largely by radiation from the sun. Incoming solar
radiation may be reflected, absorbed by land surface and water bodies,
transformed (as in photosynthesis), or emitted from the land surface as
longwave radiation. Each of these processes influences climate through
changes to temperature, winds, the water cycle, and more. The overall
process is best understood by considering the Earth's energy budget.

\begin{tcolorbox}[enhanced jigsaw, colback=white, titlerule=0mm, left=2mm, leftrule=.75mm, breakable, colbacktitle=quarto-callout-note-color!10!white, coltitle=black, rightrule=.15mm, title={The Earth's Energy Budget}, bottomtitle=1mm, toptitle=1mm, opacityback=0, toprule=.15mm, bottomrule=.15mm, colframe=quarto-callout-note-color-frame, arc=.35mm, opacitybacktitle=0.6]

The Earth's climate is driven by the sun. The balance between incoming
and outgoing radiation---Earth's radiation or energy budget---determines
the energy available for changes in temperature, precipitation, and
winds and, hence, influences atmospheric chemistry and the hydrologic
cycle. The Earth's surface, atmosphere, and clouds absorb a portion of
incoming solar radiation, thereby increasing temperatures. Energy as
longwave radiation (heat) is re-emitted to the atmosphere, clouds, or
space, thereby reducing temperatures at the source. If the absorbed
solar radiation and emitted heat are in balance, the Earth's temperature
remains constant.

\begin{figure}[H]

{\centering \includegraphics{./assets/energy-budget.jpg}

}

\caption{\label{fig-budget}The Earth's radiation balance is the main
driver of our climate. Image courtesy of National Aeronautics and Space
Administration (\textsuperscript{2})}

\end{figure}

\end{tcolorbox}

Natural factors contributing to past climate change are well documented
and include changes in atmospheric chemistry, ocean circulation
patterns, solar radiation intensity, snow and ice cover, Earth's orbital
cycle around the sun, continental position, and volcanic eruptions.
While these natural factors are linked to past climate change, they are
also incorporated in the analysis of current climate change.

Since the Industrial Revolution, global climate has changed faster than
at any other time in Earth's history (Mann et al.~1999). This rapid rate
of change---often referred to as human-caused climate change---has
resulted from changes in atmospheric chemistry, specifically increases
in greenhouse gases due to increased combustion of fossil fuels,
land-use change (e.g., deforestation), and fertilizer production (Figure
2-1) (Forster et al.~2007). The primary greenhouse gases in the Earth's
atmosphere are carbon dioxide (CO2), methane (CH4), nitrous oxide (N2O),
water vapor (H2O), and ozone (O3).

\begin{figure}

{\centering \includegraphics{./assets/ghg-emissions.png}

}

\caption{\label{fig-ghg}Changes in important global atmospheric
greenhouse gas concentrations from year 0 to 2005 AD (ppm, ppb = parts
per million and parts per billion, respectively) (Forster et al.~2007).}

\end{figure}

Incoming solar radiation is either absorbed, reflected, or re-radiated
from the Earth's surface. Since greenhouse gas concentrations are
greatest near the surface, a large fraction of this reflected and
re-radiated energy is absorbed in the lower portions of the atmosphere
(hence the increase in surface temperatures and the term ``greenhouse
effect''---see sidebar). For the total energy budget to balance, the
energy (and temperature) at the top of the atmosphere must decrease to
account for the increase of energy (and temperature) near the Earth's
surface.

At natural levels, greenhouse gases are crucial for life on Earth; they
help keep average global temperatures above freezing and at levels that
sustain plant and animal life. However, at the increased levels seen
since the Industrial Revolution (roughly 275 ppm then, 400 ppm now;
Figure~\ref{fig-ghg}), greenhouse gases are contributing to the rapid
rise of our global average temperatures by trapping more heat, often
referred to as human-caused climate change. In the following chapters,
we will refer to the impacts and effects of climate change as a result
of both natural variability and human-caused climate change.

\begin{tcolorbox}[enhanced jigsaw, colback=white, titlerule=0mm, left=2mm, leftrule=.75mm, breakable, colbacktitle=quarto-callout-note-color!10!white, coltitle=black, rightrule=.15mm, title={The Greenhouse Effect}, bottomtitle=1mm, toptitle=1mm, opacityback=0, toprule=.15mm, bottomrule=.15mm, colframe=quarto-callout-note-color-frame, arc=.35mm, opacitybacktitle=0.6]

The Earth's climate is driven by the sun. The high temperature of the
sun results in the emission of high energy, shortwave radiation. About
31\% of the shortwave radiation from the sun is reflected back to space
by clouds, air molecules, dust, and lighter colored surfaces on the
earth. Another 20\% of the shortwave radiation is absorbed by ozone in
the upper atmosphere and by clouds and water vapor in the lower
atmosphere. The remaining 49\% is transmitted through the atmosphere to
the land surfaces and oceans and is absorbed. The Earth's surface
re-emits about 79\% of the absorbed energy as longwave radiation. Unlike
shortwave radiation, the Earth's atmosphere absorbs approximately 90\%
of the longwave radiation emitted from objects on its surface. This
results because of the presence of gases such as water vapor, carbon
dioxide (CO2), methane (CH4), nitrous oxide (N2O), and various
industrial products (e.g.~chlorofluorocarbons; CFCs) that more
effectively absorb longwave radiation. In turn, the energy absorbed by
these gases is reradiated in all directions. The portion that is
redirected back towards the surface contributes to warming and a
phenomenon known as the greenhouse effect.

\begin{figure}[H]

{\centering \includegraphics{./assets/greenhouse-effect.jpg}

}

\caption{\label{fig-greenhouse}Climate change occurs when the Earth's
energy budget is not in balance. Such change generally takes place over
centuries and millennia. Human-caused climate change has been occurring
over the last 200 yr, largely because of the combustion of fossil fuels
and subsequent increase of atmospheric CO2. Carbon dioxide, as well as
CH4 and other gases, absorb and re-emit longwave radiation back to the
earth's surface that would otherwise radiate rapidly into outer space,
thus warming the Earth. This increase in incoming longwave radiation is
the greenhouse effect. Image courtesy the National Academies of Sciences
(NAS undated).}

\end{figure}

\end{tcolorbox}

\bookmarksetup{startatroot}

\hypertarget{sec-assessment}{%
\chapter{\texorpdfstring{\textbf{Draft} Climate Change
Assessments}{Draft Climate Change Assessments}}\label{sec-assessment}}

A growing awareness of our changing global climate since the 1950s has
led to a substantial body of research. For example, the National Academy
of Sciences (NAS 2011) report, American's Climate Choices, stated:

\begin{quote}
Climate change is occurring, is very likely caused primarily by human
activities, and poses significant risks to humans and the environment.
These risks indicate a pressing need for substantial action to limit the
magnitude of climate change and to prepare for adapting to its impacts.
\end{quote}

In 1990, the United Nations tasked the Intergovernmental Panel on
Climate Change (IPCC, see sidebar) with assessing existing research on
climate change. Since then, five IPCC assessments have increased our
scientific understanding of, and certainty about, global climate change.
As described later in this chapter, the assessments have incorporated
increasingly sophisticated models and analyses that consider both
natural and human contributions to changes in our climate system.

In its most recent Fifth Assessment Report, the IPCC raised the
likelihood of changes in several global climate events to ``virtually
certain'' (i.e., 99-100\% likelihood). Examples of these events include:
more frequent hot days, less frequent cold days, reductions in
permafrost, and sea-level rise (IPCC 2014).

\begin{tcolorbox}[enhanced jigsaw, colback=white, titlerule=0mm, left=2mm, leftrule=.75mm, breakable, colbacktitle=quarto-callout-note-color!10!white, coltitle=black, rightrule=.15mm, title={What is the IPCC}, bottomtitle=1mm, toptitle=1mm, opacityback=0, toprule=.15mm, bottomrule=.15mm, colframe=quarto-callout-note-color-frame, arc=.35mm, opacitybacktitle=0.6]

The Intergovernmental Panel on Climate Change is the leading
international body for the assessment of climate change. It was
established in 1988 by the United Nations Environment Programme and the
World Meteorological Organization, and subsequently endorsed by the
United Nations General Assembly. The goal of the IPCC is to provide the
world with a clear scientific view on the current state of knowledge in
climate change and its potential environmental and socioeconomic
impacts.

\begin{figure}[H]

{\centering \includegraphics{./assets/ipcc.jpg}

}

\caption{\label{fig-ipcc}The IPCC}

\end{figure}

\end{tcolorbox}

Recently, the third National Climate Assessment, produced in
collaboration with the US Global Change Research Program, provided
further insight into the anticipated climate changes for the
conterminous US. The National Climate Assessment (NCA 2014) states:

\begin{quote}
Evidence for changes in Earth's climate can be found from the top of the
atmosphere to the depths of the oceans. Researchers from around the
world have compiled this evidence using satellites, weather balloons,
thermometers at surface stations, and many other types of observing
systems that monitor the Earth's weather and climate. The sum total of
this evidence tells an unambiguous story: the planet is warming.
\end{quote}

\bookmarksetup{startatroot}

\hypertarget{sec-historical}{%
\chapter{\texorpdfstring{\textbf{Draft} Historical
Climate}{Draft Historical Climate}}\label{sec-historical}}

\hypertarget{climate-conditions-1991---2020}{%
\section{Climate Conditions 1991 -
2020}\label{climate-conditions-1991---2020}}

To assess Montana's current climate, we analyzed climate variable data
(see sidebar) from the NClimGrid climate dataset. In this section, we
review average temperature and precipitation conditions from 1991-2020
as an indicator of current climate conditions.

\begin{tcolorbox}[enhanced jigsaw, colback=white, titlerule=0mm, left=2mm, leftrule=.75mm, breakable, colbacktitle=quarto-callout-note-color!10!white, coltitle=black, rightrule=.15mm, title={Climate Variables}, bottomtitle=1mm, toptitle=1mm, opacityback=0, toprule=.15mm, bottomrule=.15mm, colframe=quarto-callout-note-color-frame, arc=.35mm, opacitybacktitle=0.6]

In analyses of climate, scientists employ a suite of 50 essential
climate variables to unify discussions (Global Climate Observing System
undated). For this assessment, we primarily focus on just two: how
climate change will affect Montana's temperature and precipitation in
the future.

\begin{itemize}
\item
  Temperature is an objective measure of how hot or cold and object is
  with reference to some standard value. Temperature differences across
  the Earth result primarily from regional differences in absorbed solar
  radiation. Seasonal variations in temperature result from the tilt of
  the Earth's axis as it rotates around the sun.
\item
  Precipitation is the quantity of water (solid or liquid) falling to
  the Earth's surface at a specific place during a given period. Like
  temperature, precipitation varies seasonally and from place to place.
  Precipitation amounts can have a dramatic impact on local
  environmental conditions, such as abundance of wildlife or potential
  for crop production.
\end{itemize}

\end{tcolorbox}

\hypertarget{temperature}{%
\subsection{Temperature}\label{temperature}}

Table~\ref{tbl-historical-temp} shows the average seasonal temperature
variation across Montana's 7 Climate Divisions from 1981-2010.
Temperatures vary across Montana, with average annual values ranging
from 29.1°F to 47.4°F across the region.

\hypertarget{tbl-historical-temp}{}
\begin{longtable}[]{@{}
  >{\raggedright\arraybackslash}p{(\columnwidth - 8\tabcolsep) * \real{0.1556}}
  >{\raggedright\arraybackslash}p{(\columnwidth - 8\tabcolsep) * \real{0.2111}}
  >{\raggedright\arraybackslash}p{(\columnwidth - 8\tabcolsep) * \real{0.2111}}
  >{\raggedright\arraybackslash}p{(\columnwidth - 8\tabcolsep) * \real{0.2111}}
  >{\raggedright\arraybackslash}p{(\columnwidth - 8\tabcolsep) * \real{0.2111}}@{}}
\caption{\label{tbl-historical-temp}Average (minimum / average /
maximum) temperatures (°F) for the seven Montana climate divisions from
1981-2010.}\tabularnewline
\toprule()
\begin{minipage}[b]{\linewidth}\raggedright
Division
\end{minipage} & \begin{minipage}[b]{\linewidth}\raggedright
Winter
\end{minipage} & \begin{minipage}[b]{\linewidth}\raggedright
Spring
\end{minipage} & \begin{minipage}[b]{\linewidth}\raggedright
Summer
\end{minipage} & \begin{minipage}[b]{\linewidth}\raggedright
Fall
\end{minipage} \\
\midrule()
\endfirsthead
\toprule()
\begin{minipage}[b]{\linewidth}\raggedright
Division
\end{minipage} & \begin{minipage}[b]{\linewidth}\raggedright
Winter
\end{minipage} & \begin{minipage}[b]{\linewidth}\raggedright
Spring
\end{minipage} & \begin{minipage}[b]{\linewidth}\raggedright
Summer
\end{minipage} & \begin{minipage}[b]{\linewidth}\raggedright
Fall
\end{minipage} \\
\midrule()
\endhead
Central & 16.8 / 27.6 / 38.5 & 37.2 / 49.9 / 62.5 & 46.7 / 61.5 / 76.3 &
22 / 32.8 / 43.6 \\
North Central & 13.6 / 24.8 / 36.1 & 38.2 / 51.3 / 64.4 & 47.2 / 62.2 /
77.2 & 19.8 / 31.2 / 42.5 \\
Northeastern & 10.7 / 21.7 / 32.8 & 39.9 / 53.5 / 67 & 49.9 / 64.9 /
79.9 & 18.3 / 29.7 / 41.1 \\
South Central & 17.1 / 28 / 38.9 & 38 / 50.7 / 63.4 & 48 / 62.8 / 77.5 &
22.2 / 33 / 43.8 \\
Southeastern & 14.3 / 26.1 / 37.9 & 40.4 / 54 / 67.5 & 50.9 / 66.4 /
81.8 & 20.3 / 32.5 / 44.7 \\
Southwestern & 15 / 24.5 / 34.1 & 33.1 / 45 / 56.9 & 42.9 / 57.1 / 71.4
& 19.6 / 29 / 38.4 \\
Western & 19 / 27.2 / 35.5 & 35.2 / 47 / 58.9 & 43.9 / 58 / 72.2 & 23 /
30.7 / 38.3 \\
\bottomrule()
\end{longtable}

Seasonal temperatures across Montana vary, with seasonal average Winter,
Spring, Summer and Fall temperatures of 25.7°F, 50.3°F, 61.9°F, and
31.3°F, respectively.

\hypertarget{precipitation}{%
\subsection{Precipitation}\label{precipitation}}

Table~\ref{tbl-historical-pr} shows the seasonal variation of
precipitation across Montana's 7 Climate Divisions from 1981-2010. The
average annual precipitation for Montana is 19.2.

\hypertarget{tbl-historical-pr}{}
\begin{longtable}[]{@{}lrrrr@{}}
\caption{\label{tbl-historical-pr}Average precipitation in inches for
the seven Montana climate divisions from 1981-2010.}\tabularnewline
\toprule()
Division & Winter & Spring & Summer & Fall \\
\midrule()
\endfirsthead
\toprule()
Division & Winter & Spring & Summer & Fall \\
\midrule()
\endhead
Central & 2.7 & 7.6 & 4.7 & 3.1 \\
North Central & 2.1 & 6.4 & 4.3 & 2.4 \\
Northeastern & 1.2 & 5.9 & 4.6 & 1.8 \\
South Central & 3.1 & 7.3 & 4.1 & 3.6 \\
Southeastern & 1.6 & 6.5 & 4.2 & 2.2 \\
Southwestern & 4.7 & 8.2 & 4.6 & 4.9 \\
Western & 8.9 & 8.8 & 4.9 & 9.5 \\
\bottomrule()
\end{longtable}

\hypertarget{historical-trends-1951---present}{%
\section{Historical Trends 1951 -
Present}\label{historical-trends-1951---present}}

We evaluated how temperature and precipitation have historically
changed, dating back to 1950. This review of historical trends helps us
provide context for future climate change scenarios explored in later
sections of this chapter. In addition, evaluating these trends can help
us better understand a) how Montana has previously experienced and
responded to changing climate, b) if projections of future change reveal
a different climate than we have previously experienced, and c) the
potential impacts of that projected change.

The presentation of trends that follows is confined to the period from
1950--2015 using data from NOAA's NClimGrid Dataset
(\textsuperscript{3}). This is widely acknowledged as the benchmark
period in climate analysis (\textsuperscript{4}), a period when our
network of meteorological sensors becomes more accurate and sufficiently
dense.

\hypertarget{temperature-1}{%
\subsection{Temperature}\label{temperature-1}}

Table~\ref{tbl-temp-trends} shows the decadal rate of change from
1950-2015 for average annual temperatures across Montana's 7 Climate
Divisions. We provide that rate of change both annually and by season
for Montana. We also present the average annual and average seasonal
changes across the region. To account partially for autocorrelation we
considered trends as significant with a conservative p value at
p\textless0.05.

\hypertarget{tbl-temp-trends}{}
\begin{longtable}[]{@{}lrrrrr@{}}
\caption{\label{tbl-temp-trends}Decadal rate of change for annual
average temperatures in °F (°C) for each region in the study area from
1950-2015. A value of 0 indicates no statistically significant change
between decadal averages.}\tabularnewline
\toprule()
Division & Annual & Winter & Spring & Summer & Fall \\
\midrule()
\endfirsthead
\toprule()
Division & Annual & Winter & Spring & Summer & Fall \\
\midrule()
\endhead
Central & 0.31 & 0.50 & 0.56 & 0.26 & 0.24 \\
North Central & 0.35 & 0.70 & 0.58 & 0.27 & 0.24 \\
Northeastern & 0.29 & 0.61 & 0.58 & 0.22 & 0.21 \\
South Central & 0.33 & 0.44 & 0.58 & 0.33 & 0.30 \\
Southeastern & 0.26 & 0.52 & 0.51 & 0.17 & 0.24 \\
Southwestern & 0.26 & 0.27 & 0.57 & 0.27 & 0.24 \\
Western & 0.30 & 0.33 & 0.47 & 0.38 & 0.30 \\
Statewide & 0.30 & 0.50 & 0.55 & 0.27 & 0.25 \\
\bottomrule()
\end{longtable}

The rate of temperature change across Montana was 0.3°F/decade
(Table~\ref{tbl-temp-trends}). Across Climate Divisions average annual
minimum and maximum temperature changes ranged from 0.18-0.16°F/decade,
respectively. Between 1950 and 2015, Montana's average annual
temperature has increased by 1.92°F; annual maximum and minimum
temperatures have increased by approximately 1.15°F and 1.02°F,
respectively.

\begin{figure}

{\centering \includegraphics{./historical_files/figure-pdf/fig-temp-trends-1.pdf}

}

\caption{\label{fig-temp-trends}Trends in annual average temperature
across each Climate Division in Montana}

\end{figure}

\hypertarget{precipitation-1}{%
\subsection{Precipitation}\label{precipitation-1}}

Table~\ref{tbl-ppt-trends} shows the decadal rate of change from
1950-2015 for average total precipitation across Montana's 7 Climate
Divisions. We provide that rate of change both annually and by season
for Montana. We also present the average annual and average seasonal
changes across the region. To account partially for autocorrelation we
considered trends as significant with a conservative p-value at
p\textless0.05.

\hypertarget{tbl-ppt-trends}{}
\begin{longtable}[]{@{}lrrrrr@{}}
\caption{\label{tbl-ppt-trends}Decadal rate of change in average
precipitation in inches/decade for the seven Montana climate divisions
from 1951-2015. A value of 0 indicates no significant
change.}\tabularnewline
\toprule()
Division & Annual & Winter & Spring & Summer & Fall \\
\midrule()
\endfirsthead
\toprule()
Division & Annual & Winter & Spring & Summer & Fall \\
\midrule()
\endhead
Central & 0.00 & -0.12 & 0.08 & 0 & 0.06 \\
North Central & 0.00 & -0.09 & 0.08 & 0 & 0.09 \\
Northeastern & 0.36 & 0.00 & 0.21 & 0 & 0.07 \\
South Central & 0.00 & 0.00 & 0.09 & 0 & 0.07 \\
Southeastern & 0.37 & 0.00 & 0.31 & 0 & 0.09 \\
Southwestern & 0.00 & -0.13 & 0.09 & 0 & 0.07 \\
Western & -0.36 & -0.56 & 0.00 & 0 & 0.00 \\
Statewide & 0.00 & -0.15 & 0.13 & 0 & 0.06 \\
\bottomrule()
\end{longtable}

The rate of precipitation change across Montana was 0in./decade
(Table~\ref{tbl-ppt-trends}). Between 1950 and 2015, Montana's average
annual precipitation has not changed by 0 inches.

\hypertarget{extreme-aspects-of-montanas-climate-in-progress}{%
\subsection{\texorpdfstring{Extreme aspects of Montana's climate
\textbf{IN
PROGRESS}}{Extreme aspects of Montana's climate IN PROGRESS}}\label{extreme-aspects-of-montanas-climate-in-progress}}

Along with analyzing historical trends in temperature and precipitation,
we performed an analysis of changes in extreme climate events since the
middle of last century. Two examples of climate extremes include periods
of intense warm or cool temperatures and significant wet or dry spells
across seasons. Because these events affect every aspect of our society,
decision makers and stakeholders are increasingly in need of historical
evaluations of extreme events and how they are changing from seasons to
centuries. The coldest temperature ever observed in the conterminous US
was -70°F at Rogers Pass outside of Helena on January 20, 1954. Since
1950, however, our analysis shows the average winter temperature has
increased by \textbf{0.4°F/decade} across the state, with an overall
average winter temperature increase of \textbf{3.6°F}. Average spring
temperatures have increased by \textbf{2.6°F} during the same period,
and average summer temperatures have risen by \textbf{2.0°F}. Montana's
fall average temperatures have increased by \textbf{1.6°F} since 1951.

We performed our analysis of climate extremes using the CLIMDEX project
(CLIMDEX undated), which provides a collection of global and regional
climate data from multiple sources. CLIMDEX is developed and maintained
by researchers at the Climate Change Research Centre and the University
of New South Wales, in collaboration with the University of Melbourne,
Climate Research Division of Environment Canada, and NOAA's National
Centers for Environmental Information. The CLIMDEX project aims to
produce a global dataset of standardized indices representing the
extreme aspects of climate. Particular attention was placed on the
changes in variables such as consecutive dry days, days of heavy
precipitation, growing season length, frost days, number of cool days
and nights, and the number of warm days and nights. Extreme
precipitation events across the United States have increased in both
intensity and frequency since 1901 (NCA 2014), including across both the
High Plains and the northwestern US (many states combined), where
studies have shown an increase in the number of days with extreme
precipitation (NCA 2014). However, for our analysis at the state level
we found no evidence of changes in extreme precipitation so it is not a
variable of focus. Here, we report those variables that did change
significantly (p\textless0.05) for Montana and, for perspective, the
climate normals for these extremes for the periods 1951--1980 and from
1981-2010 (Table 2-5).

\textbf{extremes table}

The annual number of cool days and the number of days with frost are
decreasing across Montana. We use the CLIMDEX definition of cool days as
the percentage of days when maximum temperature is lower than 10\% of
the historical observations. Coincident with warming temperatures, the
number of cool days each year during the period from 1951--2010 has
decreased by 13.3 days. Along with this trend, the number of days in
which the minimum temperatures are below 32°F (0°C; i.e.~frost days) has
decreased by 12 days during this time period. These trends have
contributed to an overall increase in the growing season length of 12
days between 1951 and 2010. In addition, the number of warm days, where
maximum temperature exceeds 90°F (32°C) based on historical conditions,
has increased by 11 days over this period. At a sub-annual level,
monthly maximum and minimum temperatures have also changed. These are
defined as the monthly maximum (minimum) value of daily maximum
(minimum) temperatures. Monthly minimum values of daily minimum
temperatures have increased by 5°F (2.8°C) from the period 1951--2010.
Over the same time period, monthly minimum values of daily maximum
temperatures have increased by 1.1°F (0.6°C).

There has been an increase in the number of warm nights and a related
decrease in the number of cool nights across Montana. We use the CLIMDEX
definition of warm nights (and cool nights) as the number of days when
minimum temperature is higher (lower) than a specified maximum (minimum)
threshold defined by historical conditions. The number of warm nights
has increased by 11 days from 1951 to 2010. The number of cool nights
has decreased by 12 days over this same period. These trends are in
agreement with observations across many portions of the continental US
(Davy and Esau 2016).

\begin{quote}
Between 1951 and 2010, the growing season in Montana increased 12 days.
\end{quote}

\begin{tcolorbox}[enhanced jigsaw, colback=white, titlerule=0mm, left=2mm, leftrule=.75mm, breakable, colbacktitle=quarto-callout-note-color!10!white, coltitle=black, rightrule=.15mm, title={Drought}, bottomtitle=1mm, toptitle=1mm, opacityback=0, toprule=.15mm, bottomrule=.15mm, colframe=quarto-callout-note-color-frame, arc=.35mm, opacitybacktitle=0.6]

Drought is a recurrent climate event that may vary in intensity and
persistence by region. Drought can have broad and potentially
devastating environmental and economic impacts (Wilhite 2000); thus, it
is a topic of ongoing, statewide concern.

Through time, Montana's people, agriculture, and industry, like its
ecosystems, have evolved with drought. Today, many entities across the
state address drought, including private and non-profit organizations,
state and federal agencies, and landowners, as well as unique watershed
partnerships.

Drought is a complex phenomenon driven by both climate, but also
affected by human-related factors (e.g., land use, water use). Although
the definition of drought varies in different operational contexts, most
definitions include several interrelated components, including:

\begin{itemize}
\item
  meteorological drought, defined as a deficit in precipitation and
  above average evapotranspiration that lead to increased aridity;
\item
  hydrological drought, characterized by reduced water levels in
  streams, lakes, and aquifers following prolonged periods of
  meteorological drought;
\item
  ecological drought, defined as a prolonged period over which an
  ecosystem's demand for water exceeds the supply (the resulting water
  deficit, or shortage, creates multiple stresses within and across
  ecosystems); and
\item
  agricultural drought, commonly understood as a deficit in soil
  moisture and water supply that lead to decreased productivity (in this
  assessment, we will treat this form of drought as an important
  component of ecological drought).
\end{itemize}

While the subsequent chapters dealing with water, agriculture, and
forests treat the subject of drought differently, each describes drought
within the context of one or more of the four definitions described
above.

\end{tcolorbox}

\bookmarksetup{startatroot}

\hypertarget{draft-teleconnections}{%
\chapter{\texorpdfstring{\textbf{Draft}
Teleconnections}{Draft Teleconnections}}\label{draft-teleconnections}}

When we think of weather, we generally think about what is happening
around us at that moment. However, the Earth's atmosphere, oceans, and
landmasses make up a continuous system, and what we experience as
weather---and also in expanded time frames as climate---is actually a
small part of much larger patterns of atmospheric circulation that
determine movements of air, moisture, and energy across the planet.
Atmospheric circulation takes on recurring patterns that link the
weather and climate across distant parts of the globe. Scientists call
these recurring or persistent patterns, teleconnections. Teleconnections
thus are climate oscillations that link across vast geographical areas
and can last for weeks to decades.

In the past, scientists identified teleconnections by observing patterns
in historical climate and weather data, and then investigating the
underlying processes driving those patterns. As global climate changes,
the nature of these connections is changing, as well. We can no longer
rely only on historical observations to understand future
teleconnections. Thus, predicting climate-related changes in
teleconnections and the impact of those changes on local weather and
climate are important areas of ongoing research.

Scientists recognize many teleconnections. We describe two of the most
important teleconnections for Montana below, the El Niño-Southern
Oscillation and the Pacific Decadal Oscillation. It is important to bear
in mind that teleconnections are happening continually, and superimposed
on each other as well as upon other long-term climate patterns. As such,
teleconnections may mask the trend of a longer-term climate signal or
enhance the signal making it appear stronger than it is. Additionally,
teleconnections can be helpful in identifying likely seasonal and annual
weather patterns and, in some cases, longer-term climate trends.

\hypertarget{el-niuxf1o-southern-oscillation}{%
\section{El Niño-Southern
Oscillation}\label{el-niuxf1o-southern-oscillation}}

The El Niño-Southern Oscillation cycle refers to a fluctuation between
unusually warm (El Niño) and cold (La Niña) waters in the tropical
Pacific, with associated changes in atmospheric circulation (the
Southern Oscillation) (Figure~\ref{fig-elnino}). El Niño and La Niña
events typically develop over 2-7 yr. During El Niño events, western
North America experiences greater flows of maritime air and reduced
flows of cold polar air from Canada. Generally drier and warmer
conditions result in the northwestern US (NWSa undated). In Montana, El
Niño winters receive roughly 70-90\% of normal precipitation, and both
winter and summer are warmer than average (Figure~\ref{fig-elnino} and
Figure~\ref{fig-elnino-anomaly}) (NWSb undated; Higgins et al.~2007).
The effects of La Niña events are generally opposite those of El Niño.
The northwestern US, including Montana, experiences increased
precipitation and cooler temperatures, while the southern states are
drier and warmer during La Niña events.

\begin{figure}

{\centering \includegraphics{./assets/el_nino.png}

}

\caption{\label{fig-elnino}Typical January-March weather anomalies and
atmospheric circulation during El Niño (top) and La Niña (bottom)
events. Image courtesy National Weather Service (NWSa undated).}

\end{figure}

\begin{figure}

{\centering \includegraphics{./assets/el_nino_anomaly.png}

}

\caption{\label{fig-elnino-anomaly}(A) Top two images show the average
anomaly in Montana's winter precipitation (left) and temperature (right)
during La Niña events. (B) Bottom two images show the average anomaly in
Montana's winter precipitation (left) and temperature (right) during El
Niño events. For Montana, El Niño winters are generally drier and
warmer; La Niña winters are generally wetter and colder. This analysis
was done using data from Livneh et al.~(2013) and is based on the study
period of 1915-2013.}

\end{figure}

\hypertarget{pacific-decadal-oscillation}{%
\section{Pacific Decadal
Oscillation}\label{pacific-decadal-oscillation}}

The Pacific Decadal Oscillation is a pattern of ocean-atmospheric
climate variability across the mid-latitude Pacific Ocean. The
oscillation varies in time from interannual to inter-decadal, with the
strongest cycle typically occurring about every 30 yr. Effects of the
Pacific Decadal Oscillation are not as intense as the El Niño-Southern
Oscillation cycle (Mantua and Hare 2002). During its warm phase, winter
temperatures are warmer throughout Alaska, western Canada, and the
western US (by an average of 2°F), and precipitation is decreased
(Figure~\ref{fig-pdo-anomaly}). Effects during the cool phase reverse,
with cooler winter temperatures and increased precipitation experienced
over western North America.

\begin{figure}

{\centering \includegraphics{./assets/el_nino_anomaly.png}

}

\caption{\label{fig-pdo-anomaly}(A) Top two images show the average
anomaly in Montana's winter precipitation (left) and temperature (right)
during the cool phase of the Pacific Decadal Oscillation. (B) Bottom two
images show the average anomaly in Montana's winter precipitation (left)
and temperature (right) during the warm phase of the Pacific Decadal
Oscillation. For Montana, the warm phase of the Pacific Decadal
Oscillation is generally associated with warmer and drier winters. Cool
phase Pacific Decadal Oscillation winters are generally wetter and
colder. This analysis was done using data from Livneh et al.~(2013) and
is based on the study period of 1915-2013.}

\end{figure}

The Pacific Decadal Oscillation and El Niño-Southern Oscillation
teleconnections may reinforce or moderate each other, depending on if
their phases are in alignment or opposition.

\bookmarksetup{startatroot}

\hypertarget{sec-future}{%
\chapter{\texorpdfstring{\textbf{Draft} Future
Projections}{Draft Future Projections}}\label{sec-future}}

\hypertarget{global-climate-modeling}{%
\section{Global Climate Modeling}\label{global-climate-modeling}}

Projecting future climate on a global scale requires modeling many
intricate relationships between the land, ocean, and atmosphere. Many
global climate and Earth system models exist, each varying in
complexity, capabilities, and limitations.

Consider one of the simplest forms of a model used for future
projections, a linear regression model. With this model, researchers
would plot a climate variable (e.g., temperature) over time, draw a
best-fit, straight line through the data, and then extend the line into
the future. That line, then, provides a means of projecting future
conditions. Whether or not those projections are valid is a separate
question. For example, the model may be based on false assumptions: the
relationship may a) not be constant through time, b) not include outside
influences such as human interventions (e.g., policy regulations), and
c) not consider system feedbacks that might enhance or dampen the
relationship being modeled.

While the linear regression model provides an instructive visual aid for
considering modeling, it is too simple for looking at climate changes,
in which the interactions are complex and often nonlinear. For example,
if temperatures rises, evaporation is expected to increase. At the same
time, increasing temperatures increase the atmosphere's capacity to hold
water. Water is a greenhouse gas so more water in the atmosphere means
the atmosphere can absorb more heat\ldots{} thus driving more
evaporation. What seemed a simple relationship has changed (possibly
dramatically) because of this feedback between temperature, evaporation,
and the water-holding capacity of the atmosphere.

Linear models do not account for such nonlinear relationships. Instead,
climate scientists account for nonlinearity through computer simulations
that describe the physical and chemical interactions between the land,
oceans, and atmosphere. These simulations, which project climate change
into the future, are called general circulation models (GCMs; see
sidebar)

\begin{tcolorbox}[enhanced jigsaw, colback=white, titlerule=0mm, left=2mm, leftrule=.75mm, breakable, colbacktitle=quarto-callout-note-color!10!white, coltitle=black, rightrule=.15mm, title={General Circulation Models}, bottomtitle=1mm, toptitle=1mm, opacityback=0, toprule=.15mm, bottomrule=.15mm, colframe=quarto-callout-note-color-frame, arc=.35mm, opacitybacktitle=0.6]

General circulation models (GCMs) help us project future climate
conditions. They are the most advanced tools currently available for
simulating the response of the global climate system---including
processes in the atmosphere, ocean, cryosphere, and land surface---to
increasing greenhouse gas concentrations.

GCMs depict the climate using a 3-D grid over the globe, typically
having a horizontal resolution of between 250 and 600 km (160 and 370
miles), 10-20 vertical layers in the atmosphere and sometimes as many as
30 layers in the oceans. Their resolution is quite coarse. Thus, impacts
at the scale of a region, for example for Montana, require downscaling
the results from the global model to a finer spatial grid (discussed
later) (text adapted from IPCC 2013b).

\end{tcolorbox}

Because of the complexities involved, climate scientists rarely rely on
a single model, but instead use an ensemble (or suite) of models. Each
model in an ensemble represents a single description of future climate
based on specific initial conditions and assumptions. The use of
multiple models helps scientists explore the variability of future
projections (i.e., how certain are we about the projection) and
incorporate the strengths, as well as uncertainties, of multiple
approaches.

For the work of the Montana Climate Assessment, we employed an ensemble
from the sixth iteration of the Coupled Model Intercomparison Project
(CMIP6), which includes over 100 GCMs depending on the experiment
conducted (\textsuperscript{5}). The World Climate Research Program
describes CMIP as ``a standard experimental protocol for studying the
output'' of GCMs (\textsuperscript{6}). It provides a means of
validating, comparing, documenting, and accessing diverse climate model
results. The CMIP project dates back to 1995, with the sixth iteration
(CMIP6) starting in 2016 and providing climate data for the latest IPCC
Sixth Assessment Report (\textsuperscript{7}).

We employed 8 individual GCMs from the CMIP6 project for the Montana
Climate Assessment ensemble, chosen because they provide daily outputs
and are found to have a realistic performance over North America
(\textsuperscript{8})

The benefits of using CMIP6 data are that each model in the ensemble a)
has been rigorously evaluated, and b) uses the same standard
socioeconomic trajectories---known as Shared Socioeconomic Pathways
(SSPs)---to describe future greenhouse gas emissions. ``The SSPs are
based on five narratives describing alternative socio-economic
developments, including sustainable development, regional rivalry,
inequality, fossil-fueled development, and middle-of-the-road
development'' (\textsuperscript{9})

\begin{tcolorbox}[enhanced jigsaw, colback=white, titlerule=0mm, left=2mm, leftrule=.75mm, breakable, colbacktitle=quarto-callout-note-color!10!white, coltitle=black, rightrule=.15mm, title={Shared Socioeconomic Pathways}, bottomtitle=1mm, toptitle=1mm, opacityback=0, toprule=.15mm, bottomrule=.15mm, colframe=quarto-callout-note-color-frame, arc=.35mm, opacitybacktitle=0.6]

There are five different SSP categories that climate projections are
grouped by (following text taken from\textsuperscript{9}):

\begin{itemize}
\item
  SSP1 Sustainability: Taking the Green Road (Low challenges to
  mitigation and adaptation) The world shifts gradually, but
  pervasively, toward a more sustainable path, emphasizing more
  inclusive development that respects perceived environmental
  boundaries. Management of the global commons slowly improves,
  educational and health investments accelerate the demographic
  transition, and the emphasis on economic growth shifts toward a
  broader emphasis on human well-being. Driven by an increasing
  commitment to achieving development goals, inequality is reduced both
  across and within countries. Consumption is oriented toward low
  material growth and lower resource and energy intensity.
\item
  SSP2 Middle of the Road: (Medium challenges to mitigation and
  adaptation) The world follows a path in which social, economic, and
  technological trends do not shift markedly from historical patterns.
  Development and income growth proceeds unevenly, with some countries
  making relatively good progress while others fall short of
  expectations. Global and national institutions work toward but make
  slow progress in achieving sustainable development goals.
  Environmental systems experience degradation, although there are some
  improvements and overall the intensity of resource and energy use
  declines. Global population growth is moderate and levels off in the
  second half of the century. Income inequality persists or improves
  only slowly and challenges to reducing vulnerability to societal and
  environmental changes remain.
\item
  SSP3 Regional Rivalry: A Rocky Road (High challenges to mitigation and
  adaptation) A resurgent nationalism, concerns about competitiveness
  and security, and regional conflicts push countries to increasingly
  focus on domestic or, at most, regional issues. Policies shift over
  time to become increasingly oriented toward national and regional
  security issues. Countries focus on achieving energy and food security
  goals within their own regions at the expense of broader-based
  development. Investments in education and technological development
  decline. Economic development is slow, consumption is
  material-intensive, and inequalities persist or worsen over time.
  Population growth is low in industrialized and high in developing
  countries. A low international priority for addressing environmental
  concerns leads to strong environmental degradation in some regions.
\item
  SSP4 Inequality: A Road Divided (Low challenges to mitigation, high
  challenges to adaptation) Highly unequal investments in human capital,
  combined with increasing disparities in economic opportunity and
  political power, lead to increasing inequalities and stratification
  both across and within countries. Over time, a gap widens between an
  internationally-connected society that contributes to knowledge- and
  capital-intensive sectors of the global economy, and a fragmented
  collection of lower-income, poorly educated societies that work in a
  labor intensive, low-tech economy. Social cohesion degrades and
  conflict and unrest become increasingly common. Technology development
  is high in the high-tech economy and sectors. The globally connected
  energy sector diversifies, with investments in both carbon-intensive
  fuels like coal and unconventional oil, but also low-carbon energy
  sources. Environmental policies focus on local issues around middle
  and high income areas.
\item
  SSP5 Fossil-fueled Development: Taking the Highway (High challenges to
  mitigation, low challenges to adaptation) This world places increasing
  faith in competitive markets, innovation and participatory societies
  to produce rapid technological progress and development of human
  capital as the path to sustainable development. Global markets are
  increasingly integrated. There are also strong investments in health,
  education, and institutions to enhance human and social capital. At
  the same time, the push for economic and social development is coupled
  with the exploitation of abundant fossil fuel resources and the
  adoption of resource and energy intensive lifestyles around the world.
  All these factors lead to rapid growth of the global economy, while
  global population peaks and declines in the 21st century. Local
  environmental problems like air pollution are successfully managed.
  There is faith in the ability to effectively manage social and
  ecological systems, including by geo-engineering if necessary.
\end{itemize}

\end{tcolorbox}

For the Montana Climate Assessment, we explore the SSP1, SSP2, SSP3, and
SSP5 scenarios.

Due to their complexity and global extent, GCMs can be computationally
intensive. Thus, scientists often make climate projections at coarse
spatial resolution where each projected data point is an average value
of a grid cell that measures hundreds of miles (kilometers) across.

For areas where the terrain and land cover are relatively homogenous
(e.g., an expanse of the Great Plains), such coarse grid cells may be
adequate to capture important climate processes. But in areas with
complex landscapes like Montana, data points so widely spaced are
inadequate to reflect variability in terrain and vegetation and their
influence on climate. A 100 mile (161 km) grid, for example, might not
capture the climate effects of a small mountain range rising out of the
eastern Montana plains or the climate differences between mountain
summits and valleys in western Montana where temperature and
precipitation vary greatly.

To capture such important terrain characteristics, scientist take the
coarse-resolution output from a GCM and statistically attribute the
results from those models to smaller regions at higher resolution (e.g.,
grid points at 1 mile rather than 100 mile apart). This process, called
downscaling, more accurately represents climate across smaller, more
complex landscapes, including Montana.

For this climate assessment, we used a statistical downscaling method
called the Bias-Correction Spatial Disaggregation. By using a downscaled
dataset---rather than the original output from the ensemble of GCMs---we
gained the ability to evaluate temperature and precipitation at
relatively high resolution statewide before conveying the results at the
climate division scale. Additionally, we were able to aggregate data
points within each of Montana's seven climate divisions, and look at
Montana's climate future in different geographic areas. Aggregating to
the climate-division level minimizes the potential for false precision
by reporting results at spatial scales that better represent underlying
climate processes.

The 8 downscaled GCMs in CMIP6 were evaluated at two future time
periods: 1) mid century (2040--2069) and 2) end-of-century (2070--2099).
Thirty-year averages of these future projections were then compared to a
historical (1991--2020) 30-year average, which results in a projected
difference, or change, from historical conditions. We make those
projections using the stabilization the four SSPs outlined above. These
future projections were then compared to the historical trends in
Montana to reveal the major climate-associated changes that Montana is
likely to experience in the future.

\hypertarget{summary-of-projections}{%
\section{Summary of Projections}\label{summary-of-projections}}

\hypertarget{temperature-summary}{%
\subsection{Temperature Summary}\label{temperature-summary}}

In general, there is high model agreement and low uncertainty that
temperatures and associated temperature metrics will increase both by
mid century and end-of-century. For both periods, annual and seasonal
temperature averages, the number of days/yr with extreme heat, and the
overall length of the growing season are projected to increase.
Differences exist in projections all SSP scenarios, with the lower
consequence scenarios showing lower magnitudes of change than the more
extreme.Many of the trends and spatial patterns seen in the mid-century
projections are extended and exacerbated in the end-of-century
projections. The range of model outputs also increases for
end-of-century projections, suggesting that the magnitude of change
becomes more uncertain in the models further out in time.

Regardless of uncertainties, the GCMs show full agreement regarding the
direction of change: temperatures will be increasing.

\hypertarget{precipitation-summary}{%
\subsection{Precipitation Summary}\label{precipitation-summary}}

In mid-century and end-of-century projections, average annual
precipitation and variability increase across the state, as does winter,
spring, and fall precipitation. Summers, however, show slight decreases
in precipitation. The projections suggest slight increases in both the
annual number of consecutive dry and wet days. Overall, the differences
in precipitation resulting from the different emission scenarios are
small when compared to the impact of the emission scenarios on the
temperature projections. Uncertainty in the projections generally
increases the further out in time (i.e., in the end-of-century
projections), as well as for the more extreme emission scenarios.

\hypertarget{temperature-projections}{%
\section{Temperature Projections}\label{temperature-projections}}

Below we provide projections for various aspects of Montana's future
temperature based on our modeling analysis.

We discuss a subset of our modeling results here, including a)
temperature projections reported by the mean values of the 8 GCM
ensemble and b) figures that include maps and graphs that represent the
mean value and distribution of values observed for temperature across
the 8 GCMs.

We also report a percent agreement of the 8 GCMs used for the analysis.
The percent agreement represents the number of GCMs that project the
same sign of change (i.e., positive or negative) as the mean value. For
example, if the mean value is positive and 7 out of 8 models also
project positive change, then the percent agreement would be 100 x 7/8 =
87.5\%. This simple calculation helps convey the uncertainty in the
projections.

\hypertarget{average-annual-temperatures}{%
\subsection{Average Annual
Temperatures}\label{average-annual-temperatures}}

Figure~\ref{fig-tavg-box} shows projected changes in average annual
temperatures across all Climate Divisions for both the mid- and
end-of-century. Below, projected changes in average annual temperature
across the domain and associated model agreements are given:

\begin{itemize}
\tightlist
\item
  Mid-century: Across the domain, the majority of SSP scenarios project
  that the average annual temperature in the mid-century will increase.
  Changes in average annual temperature across SSPs range from 3.6 °F
  (100\% model agreement) to 5.48 °F (100\% model agreement)
\end{itemize}

\begin{itemize}
\tightlist
\item
  End-of-century: Across the domain, the majority of SSP scenarios
  project that the average annual temperature in the end-of-century will
  increase. Changes in average annual temperature across SSPs range from
  3.65 °F (100\% model agreement) to 9.23 °F (100\% model agreement)
\end{itemize}

\begin{figure}

{\centering \includegraphics{./future_files/figure-pdf/fig-tavg-box-1.pdf}

}

\caption{\label{fig-tavg-box}Graphs showing the distribution of
projected changes in temperature (°F) projected for each Climate
Division across all SSP scenarios. The top row shows mid-century
(2040-2069) projections and the bottom row shows end-of-century
(2070-2099) projections.}

\end{figure}

\hypertarget{average-daily-minimum-temperatures}{%
\subsection{Average Daily Minimum
Temperatures}\label{average-daily-minimum-temperatures}}

Figure~\ref{fig-tmmn-map} shows spatially distributed changes in minimum
annual temperature across all Climate Divisions for both the mid- and
end-of-century. Below, projected changes in both minimum temperature and
associated model agreements are given for the entire domain:

\begin{itemize}
\tightlist
\item
  Mid-century: Across the domain, the majority of SSP scenarios project
  that the average annual minimum temperature in the mid-century will
  increase. Changes in average annual minimum temperature across SSPs
  range from 3.54 °F (100\% model agreement) to 5.53 °F (100\% model
  agreement)
\end{itemize}

\begin{itemize}
\tightlist
\item
  End-of-century: Across the domain, the majority of SSP scenarios
  project that the average annual minimum temperature in the
  end-of-century will increase. Changes in average annual minimum
  temperature across SSPs range from 3.51 °F (100\% model agreement) to
  9.34 °F (100\% model agreement)
\end{itemize}

\begin{figure}

{\centering \includegraphics{./future_files/figure-pdf/fig-tmmn-map-1.pdf}

}

\caption{\label{fig-tmmn-map}The projected increase in annual average
daily minimum temperature (°F) for each Climate Division in Montana for
the periods 2049-2069 and 2070-2099 for all SSP scenarios}

\end{figure}

\hypertarget{average-daily-maximum-temperatures}{%
\subsection{Average Daily Maximum
Temperatures}\label{average-daily-maximum-temperatures}}

Figure~\ref{fig-tmmx-map} shows spatially distributed changes in maximum
annual temperature across all Climate Divisions for both the mid- and
end-of-century. Below, projected changes in both maxiimum temperature
and associated model agreements are given for the entire domain:

\begin{itemize}
\tightlist
\item
  Mid-century: Across the domain, the majority of SSP scenarios project
  that the average annual maximum temperature in the mid-century will
  increase. Changes in average annual maximum temperature across SSPs
  range from 3.66 °F (100\% model agreement) to 5.43 °F (100\% model
  agreement)
\end{itemize}

\begin{itemize}
\tightlist
\item
  End-of-century: Across the domain, the majority of SSP scenarios
  project that the average annual maximum temperature in the
  end-of-century will increase. Changes in average annual maximum
  temperature across SSPs range from 3.78 °F (100\% model agreement) to
  9.12 °F (100\% model agreement)
\end{itemize}

\begin{figure}

{\centering \includegraphics{./future_files/figure-pdf/fig-tmmx-map-1.pdf}

}

\caption{\label{fig-tmmx-map}The projected increase in annual average
daily maximum temperature (°F) for each Climate Division in Montana for
the periods 2049-2069 and 2070-2099 for all SSP scenarios}

\end{figure}

\hypertarget{average-monthly-temperatures}{%
\subsection{Average Monthly
Temperatures}\label{average-monthly-temperatures}}

Figure~\ref{fig-tavg-heatmap} shows projected changes in average monthly
temperatures across all Climate Divisions for both the mid- and
end-of-century. Below, projected changes in average monthly temperature
across the domain and associated model agreements are given:

\begin{itemize}
\tightlist
\item
  Mid Century: Average Temperature is projected to increase in the
  Winter, with values ranging from 3.47 °F to 5.22 °F (100\% model
  agreement) depending on the SSP scenario. Average Temperature is
  projected to increase in the Spring, with values ranging from 3.29 °F
  to 4.5 °F (100\% model agreement) depending on the SSP scenario.
  Average Temperature is projected to increase in the Summer, with
  values ranging from 3.65 °F to 6.23 °F (100\% model agreement)
  depending on the SSP scenario. Average Temperature is projected to
  increase in the Fall, with values ranging from 3.99 °F to 5.99 °F
  (100\% model agreement) depending on the SSP scenario.
\end{itemize}

\begin{itemize}
\tightlist
\item
  End of Century: Average Temperature is projected to increase in the
  Winter, with values ranging from 3.22 °F to 8.67 °F (99\% model
  agreement) depending on the SSP scenario. Average Temperature is
  projected to increase in the Spring, with values ranging from 3.17 °F
  to 7.39 °F (100\% model agreement) depending on the SSP scenario.
  Average Temperature is projected to increase in the Summer, with
  values ranging from 4.14 °F to 10.88 °F (100\% model agreement)
  depending on the SSP scenario. Average Temperature is projected to
  increase in the Fall, with values ranging from 4.06 °F to 10.02 °F
  (100\% model agreement) depending on the SSP scenario.
\end{itemize}

\begin{figure}

{\centering \includegraphics{./future_files/figure-pdf/fig-tavg-heatmap-1.pdf}

}

\caption{\label{fig-tavg-heatmap}The projected monthly increase in
average temperature (°F) for each Climate Division in Montana in the
mid-century (2040-2069) and end-of century (2070 - 2099) for all SSP
scenarios}

\end{figure}

\hypertarget{number-of-days-above-90f-32c}{%
\subsection{Number of Days Above 90°F
(32°C)}\label{number-of-days-above-90f-32c}}

Figure~\ref{fig-90-map} and Figure~\ref{fig-90-box} show projected
changes in number of days above 90°F for both the mid- and
end-of-century. Below, projected changes in days above 90°F across the
domain and associated model agreements are given:

\begin{itemize}
\tightlist
\item
  Mid-century: Across the domain, the majority of SSP scenarios project
  that the number of days above 90°F in the mid-century will increase.
  Changes in number of days above 90°F across SSPs range from 14.16 days
  (100\% model agreement) to 26.85 days (100\% model agreement)
\end{itemize}

\begin{itemize}
\tightlist
\item
  End-of-century: Across the domain, the majority of SSP scenarios
  project that the number of days above 90°F in the end-of-century will
  increase. Changes in number of days above 90°F across SSPs range from
  17.35 days (100\% model agreement) to 48.97 days (100\% model
  agreement)
\end{itemize}

\begin{figure}

{\centering \includegraphics{./future_files/figure-pdf/fig-90-map-1.pdf}

}

\caption{\label{fig-90-map}The projected increases in number of days
above 90°F (32°C) for each Climate Division in Montana over two periods
2040-2069 and 2070-2099 for all SSP scenarios}

\end{figure}

\begin{figure}

{\centering \includegraphics{./future_files/figure-pdf/fig-90-box-1.pdf}

}

\caption{\label{fig-90-box}Graphs showing the distribution across
ensemble members of the increase in number of days per year above 90°F
(32°C) projected for each Climate Division all SSP scenarios and both
mid-century (2040-2069) and end-of-century (2070-2099) projections.}

\end{figure}

\hypertarget{number-of-days-where-minimum-temperatures-are-above-32f-0c}{%
\subsection{Number of Days Where Minimum Temperatures are Above 32°F
(0°C)}\label{number-of-days-where-minimum-temperatures-are-above-32f-0c}}

Figure~\ref{fig-ff-map} and Figure~\ref{fig-ff-box} show projected
changes in number of days freeze-free days for both the mid- and
end-of-century. Below, projected changes in the number of freeze-free
days across the domain and associated model agreements are given:

\begin{itemize}
\tightlist
\item
  Mid-century: Across the domain, the majority of SSP scenarios project
  that the number of freeze-free days in the mid-century will increase.
  Changes in number of freeze-free days across SSPs range from 22.29
  days (100\% model agreement) to 32.1 days (100\% model agreement)
\end{itemize}

\begin{itemize}
\tightlist
\item
  End-of-century: Across the domain, the majority of SSP scenarios
  project that the number of freeze-free days in the end-of-century will
  increase. Changes in number of freeze-free days across SSPs range from
  21.02 days (100\% model agreement) to 55.11 days (100\% model
  agreement)
\end{itemize}

\begin{figure}

{\centering \includegraphics{./future_files/figure-pdf/fig-ff-map-1.pdf}

}

\caption{\label{fig-ff-map}The projected change in the number of
frost-free days for each Climate Division in Montana over two periods
2040-2069 and 2070-2099 for all SSP scenarios.}

\end{figure}

\begin{figure}

{\centering \includegraphics{./future_files/figure-pdf/fig-ff-box-1.pdf}

}

\caption{\label{fig-ff-box}Graphs showing the increases in frost-free
days/yr projected for each Climate Division across all SSP scenarios.
The top row shows mid-century projections (2040-2069) and the bottom row
shows end-of-century projections (2070-2099).}

\end{figure}

\hypertarget{summary}{%
\subsection{Summary}\label{summary}}

\textbf{Talk with team about how to summarize}

\hypertarget{precipitation-projections}{%
\section{Precipitation Projections}\label{precipitation-projections}}

Below we provide projections of Montana's future precipitation based on
our modeling efforts. Those projections cover all SSP scenarios and two
periods: mid century (2040-2069) and end-of-century (2070-2099).

We discuss a subset of our precipitation modeling results here,
including a) precipitation projections reported by the mean values of
the 8 GCM ensemble and b) figures that include maps and graphs that
represent the mean and distribution of values observed for precipitation
across the 8 GCMs. Special consideration is required for interpretations
of precipitation changes in Montana's complex terrain. Precipitation
increases drastically with elevation such as that found in northwest
Montana. Here, mean values do not characterize the potential for spatial
variability that exists within these regions.

\hypertarget{average-annual-precipitation}{%
\subsection{Average Annual
Precipitation}\label{average-annual-precipitation}}

Figure~\ref{fig-pr-map} and Figure~\ref{fig-pr-box} show projected
changes in annual total precipitation for both the mid- and
end-of-century. Below, projected changes in total annual precipitation
across the domain and associated model agreements are given:

\begin{itemize}
\tightlist
\item
  Mid-century: Across the domain, the majority of SSP scenarios project
  that the total annual precipitation in the mid-century will increase.
  Changes in total annual precipitation across SSPs range from 0.99
  inches (75\% model agreement) to 1.3 inches (88\% model agreement)
\end{itemize}

\begin{itemize}
\tightlist
\item
  End-of-century: Across the domain, the majority of SSP scenarios
  project that the total annual precipitation in the end-of-century will
  increase. Changes in total annual precipitation across SSPs range from
  1.06 inches (88\% model agreement) to 1.46 inches (88\% model
  agreement)
\end{itemize}

\begin{figure}

{\centering \includegraphics{./future_files/figure-pdf/fig-pr-map-1.pdf}

}

\caption{\label{fig-pr-map}The projected change in annual precipitation
(inches) for each Climate Division in Montana over two periods 2040-2069
and 2070-2099 for all SSP scenarios.}

\end{figure}

\begin{figure}

{\centering \includegraphics{./future_files/figure-pdf/fig-pr-box-1.pdf}

}

\caption{\label{fig-pr-box}Graphs showing annual precipitation change
(in inches) projected for each Climate Division for all SSP scenarios.
The top row shows mid-century projections (2040-2069) and the bottom row
shows end-of-century projections (2070-2099).}

\end{figure}

\hypertarget{interannual-variability}{%
\subsection{Interannual Variability}\label{interannual-variability}}

Interannual variability (i.e., the amount precipitation changes from
year to year) is projected to increase across the domain by mid century
and increase by the end-of-century for the majority of SSP scenarios
(Figure~\ref{fig-iv-box}). These changes could be attributed to wet
years getting wetter, dry years getting drier, or some combination of
both.

\begin{figure}

{\centering \includegraphics{./future_files/figure-pdf/fig-iv-box-1.pdf}

}

\caption{\label{fig-iv-box}Graphs showing the interannual variability of
precipitation projected for each Climate Division for all SSP scenarios.
The top row shows mid-century projections (2040-2069) and the bottom row
shows for end-of-century projections (2070-2099).}

\end{figure}

\hypertarget{monthly-and-seasonal-change-in-average-precipitation}{%
\subsection{Monthly and Seasonal Change in Average
Precipitation}\label{monthly-and-seasonal-change-in-average-precipitation}}

Figure~\ref{fig-pr-heatmap} shows projected changes in average monthly
precipitation across all Climate Divisions for both the mid- and
end-of-century. Below, projected changes in average monthly
precipitation across the domain and associated model agreements are
given:

\begin{figure}

{\centering \includegraphics{./future_files/figure-pdf/fig-pr-heatmap-1.pdf}

}

\caption{\label{fig-pr-heatmap}Projected monthly change in average
precipitation (inches) for each Climate Division in Montana in the
mid-century projections (2040-2069) for all SSP scenarios.}

\end{figure}

\begin{itemize}
\tightlist
\item
  Mid Century: Total Precipitation is projected to increase in the
  Winter, with values ranging from 0.06 inches to 0.1 inches (77\% model
  agreement) depending on the SSP scenario. Total Precipitation is
  projected to increase in the Spring, with values ranging from 0.25
  inches to 0.3 inches (91\% model agreement) depending on the SSP
  scenario. Total Precipitation is projected to decrease in the Summer,
  with values ranging from -0.05 inches to 0.07 inches (65\% model
  agreement) depending on the SSP scenario. Total Precipitation is
  projected to increase in the Fall, with values ranging from 0.01
  inches to 0.06 inches (67\% model agreement) depending on the SSP
  scenario.
\end{itemize}

\begin{itemize}
\tightlist
\item
  End of Century: Total Precipitation is projected to increase in the
  Winter, with values ranging from 0.05 inches to 0.17 inches (82\%
  model agreement) depending on the SSP scenario. Total Precipitation is
  projected to increase in the Spring, with values ranging from 0.23
  inches to 0.42 inches (90\% model agreement) depending on the SSP
  scenario. Total Precipitation is projected to decrease in the Summer,
  with values ranging from -0.2 inches to 0.04 inches (67\% model
  agreement) depending on the SSP scenario. Total Precipitation is
  projected to increase in the Fall, with values ranging from 0.03
  inches to 0.09 inches (74\% model agreement) depending on the SSP
  scenario.
\end{itemize}

\hypertarget{projected-changes-in-consecutive-dry-days}{%
\subsection{Projected Changes in Consecutive Dry
Days}\label{projected-changes-in-consecutive-dry-days}}

To assess changes in the frequency of dry events, we determined the
annual number of dry days (defined as days when precipitation is less
than 0.01 inch {[}0.03 cm{]}), then calculated the maximum number of
consecutive dry days/yr averaged over the 30-year periods of interest.
Across the domain, we found a increase in consecutive dry days for the
mid-century across the all SSPs (65.25\% agreement) and a increase in
consecutive dry days for the end-of-century across the all SSPs (75\%
agreement). Figures Figure~\ref{fig-dd-map} and Figure~\ref{fig-dd-box}
show changes in projected number of dry days across Climate Divisions
and domain-wide projections for mid- and end-of-century are given below:

\begin{itemize}
\tightlist
\item
  Mid-century: Across the domain, the majority of SSP scenarios project
  that the annual number of consecutive dry days in the mid-century will
  increase. Changes in annual number of consecutive dry days across SSPs
  range from 0.78 days (62\% model agreement) to 1.5 days (75\% model
  agreement)
\end{itemize}

\begin{itemize}
\tightlist
\item
  End-of-century: Across the domain, the majority of SSP scenarios
  project that the annual number of consecutive dry days in the
  end-of-century will increase. Changes in annual number of consecutive
  dry days across SSPs range from 1.3 days (62\% model agreement) to
  3.27 days (75\% model agreement)
\end{itemize}

\begin{figure}

{\centering \includegraphics{./future_files/figure-pdf/fig-dd-map-1.pdf}

}

\caption{\label{fig-dd-map}The projected change in the number of
consecutive dry days (\textless0.1 inch {[}0.3 cm{]} of precipitation)
for each Climate Division in Montana over two periods 2040-2069 and
2070-2099 for all SSP scenarios.}

\end{figure}

\begin{figure}

{\centering \includegraphics{./future_files/figure-pdf/fig-dd-box-1.pdf}

}

\caption{\label{fig-dd-box}Graphs showing the number of consecutive dry
days in a year projected for each Climate Division in both stabilization
(RCP4.5) and business-as-usual (RCP8.5) emission scenarios. The top row
shows mid-century projections (2040-2069) and the bottom row shows
end-of-century projections (2070-2099).}

\end{figure}

\hypertarget{projected-change-in-wet-days}{%
\subsection{Projected Change in Wet
Days}\label{projected-change-in-wet-days}}

To evaluate changes in wet events, we calculated the number of days/yr
where precipitation is greater than 1.0 inch (2.5 cm) and average those
values over the period of interest. Figures Figure~\ref{fig-wd-box}
shows changes in projected number of wet days across Climate Divisions
and domain-wide projections for mid- and end-of-century are given below:

\begin{itemize}
\tightlist
\item
  Mid-century: Across the domain, the majority of SSP scenarios project
  that the annual number of consecutive wet days in the mid-century will
  increase. Changes in annual number of consecutive wet days across SSPs
  range from 0.11 days (88\% model agreement) to 0.13 days (100\% model
  agreement)
\end{itemize}

\begin{itemize}
\tightlist
\item
  End-of-century: Across the domain, the majority of SSP scenarios
  project that the annual number of consecutive wet days in the
  end-of-century will increase. Changes in annual number of consecutive
  wet days across SSPs range from 0.12 days (100\% model agreement) to
  0.19 days (100\% model agreement)
\end{itemize}

\begin{figure}

{\centering \includegraphics{./future_files/figure-pdf/fig-wd-box-1.pdf}

}

\caption{\label{fig-wd-box}Graphs showing the increase in the number of
wet days/yr projected for each Climate Division in all SSP scenarios.
The top row shows projections for mid century (2040-2069) and the bottom
row shows projections for end-of-century (2070-2099).}

\end{figure}

\bookmarksetup{startatroot}

\hypertarget{sec-gaps}{%
\chapter{\texorpdfstring{\textbf{Draft} Key Knowledge
Gaps}{Draft Key Knowledge Gaps}}\label{sec-gaps}}

\begin{enumerate}
\def\labelenumi{\arabic{enumi}.}
\tightlist
\item
  Additional climate variables.---Our analysis provides a critical local
  look at changes for two important climate variables, precipitation and
  temperature. However, Montana's climate and its impacts go beyond
  these. A more in depth downscaling effort that involves physics based
  models will be required to evaluate two additional important
  variables, evapotranspiration and drought.
\item
  Land use and land cover change.---Most climate analyses do not account
  for changes in land cover with climatic trends. However, interactions
  between climate, vegetation cover, and land use quality are tightly
  coupled. For example, with changes in temperature and precipitation,
  ecosystems within Montana may shift to drier conditions resulting in
  changes to vegetation types. This would contribute to a difference in
  evapotranspiration rates and aridity.
\item
  Precipitation timing and form.---We took a first look at changes in
  Montana's precipitation. However, it is well known that the timing
  (winter versus spring and summer) and form (rain versus snow) of
  Montana's precipitation is critical for areas such as water, forests,
  and agriculture resources. More work that incorporates physically
  based, distributed hydrological models is required to understand how
  our precipitation distribution will change in both space (low
  elevations to mountaintops) and time.
\end{enumerate}

\bookmarksetup{startatroot}

\hypertarget{sec-conclusions}{%
\chapter{\texorpdfstring{\textbf{Draft}
Conclusions}{Draft Conclusions}}\label{sec-conclusions}}

The analysis presented in this chapter shows that statewide, Montana has
warmed---up to 1.9°F the between 1951 and 2015. Seasonally, that warming
has been greatest in winter (3.2°F) and spring (3.5°F {[}1.4°C{]}).
Montana's number of frost days has decreased by \textbf{12 days} since
1951. Statewide, average annual precipitation did not change between
1951 and 2015, although variations caused by global climate
oscillations, such as El Niño events, explain some of the historical
precipitation variability in parts of the state.

With this historical context, we considered Montana's future under four
shared socioeconomic pathways. Using those pathways, we employed
standard modeling techniques available to climate scientists
today---ensembles of general circulation models---and projected
Montana's climate over the next century. Our analyses focused on
projecting the possible range of temperature and precipitation amounts
in Montana, under our chosen greenhouse gas emission scenarios.

\begin{quote}
While the model results varied, one message is imminently clear: Montana
in the coming century will be a warmer place.
\end{quote}

In Table~\ref{tbl-summary} we provide a summary of the work done and
described in this chapter. In summary, Montana is projected to continue
to warm in all geographic locations, seasons, and under all emission
scenarios throughout the 21st century. By mid century, Montana
temperatures are projected to increase by up to 5.5°F; by the end of the
century, temperatures will increase by up to 9.2°F. Projections show
that we could have up to 55 more frost-free days at the end of the
century. Likewise, frequency of extreme heat will increase. In eastern
Montana, for example, we may have more then 60 days/yr in which maximum
temperatures exceed 90°F (32°C).

In mid- and end-of-century projections, average annual precipitation and
variability increase across the state, as do winter, spring, and fall
precipitation. Summer months, however, show small decreases in
precipitation. Current projections suggest slight increases in both dry
and wet events.

Montanans must be prepared for projected increases in temperature in the
future. Because of its interior location, Montana has warmed more over
the last 65 yr than the national average, and it will experience greater
warming than most parts of the country in the future, particularly when
compared to states in coastal regions. Key to the concern is that coming
temperature changes will be larger in magnitude and occur more rapidly
than any time since our 1889 declaration of statehood (and, to be sure,
well before).

\begin{quote}
Montana's average annual temperature is projected to increase through
the end-of-century for all models, all emission scenarios, and in all
geographic locations.
\end{quote}

\hypertarget{tbl-summary}{}
\begin{longtable}[]{@{}
  >{\raggedright\arraybackslash}p{(\columnwidth - 2\tabcolsep) * \real{0.0664}}
  >{\raggedright\arraybackslash}p{(\columnwidth - 2\tabcolsep) * \real{0.9336}}@{}}
\caption{\label{tbl-summary}Summary of climate metrics described in this
chapter.}\tabularnewline
\toprule()
\begin{minipage}[b]{\linewidth}\raggedright
Climate Metric
\end{minipage} & \begin{minipage}[b]{\linewidth}\raggedright
Trend and Future Scenario
\end{minipage} \\
\midrule()
\endfirsthead
\toprule()
\begin{minipage}[b]{\linewidth}\raggedright
Climate Metric
\end{minipage} & \begin{minipage}[b]{\linewidth}\raggedright
Trend and Future Scenario
\end{minipage} \\
\midrule()
\endhead
Atmospheric CO2 Concentrations & Global atmospheric carbon dioxide
concentrations have increased over 100 ppm since Montana statehood and
are projected to increase under both future scenarios considered
here. \\
Average Temperature & Since 1951, average statewide temperatures have
increased by 0.3°F/decade, with greatest warming in spring; projected to
increase by 3-6°F by mid century, with greatest warming in summer and
winter and in the east \\
Maximum Temperatures & Maximum temperatures have increased most in
spring and are projected to increase 4-5°F by mid century. \\
Days above 90°F & Extreme heat days are projected to increase by 14-27
additional days by mid century, with greatest increases in the northeast
and south. \\
Minimum Temperatures & Minimum temperatures are projected to increase
4-6°F by mid century. \\
Frost-free Days & Frost-free days are projected to increase by 22-32
days by mid century, particularly in the west. \\
Average Precipitation & Statewide precipitation has decreased in winter
(0.15 inches/decade) since 1951, but no significant change has occurred
in annual mean precipitation, probably because of very slight increases
in spring and fall precipitation. Precipitation is projected to
increase, primarily in spring (0.25-0.3 inches; a slight statewide
decrease in summer precipitation and increased year-to-year variability
of precipitation are projected, as well. \\
Number of Consecutive Dry Days & Projected changes range from 1.3 to 3.3
by end of the century. However, increased variability in precipitation
suggests potential for more severe droughts, particularly in connection
with climate oscillations. \\
Number of Consecutive Wet Days & Slight increases across all climate
divisions and SSP scenarios. \\
\bottomrule()
\end{longtable}

\bookmarksetup{startatroot}

\hypertarget{draft-references}{%
\chapter*{\texorpdfstring{\textbf{Draft}
References}{Draft References}}\label{draft-references}}
\addcontentsline{toc}{chapter}{\textbf{Draft} References}

\markboth{\textbf{Draft} References}{\textbf{Draft} References}

\hypertarget{refs}{}
\begin{CSLReferences}{0}{0}
\leavevmode\vadjust pre{\hypertarget{ref-cc_def}{}}%
\CSLLeftMargin{1. }%
\CSLRightInline{U.S. Global Change Research Program.
\href{https://www.globalchange.gov/climate-change/glossary}{Glossary}.
(undated).}

\leavevmode\vadjust pre{\hypertarget{ref-energy_balance}{}}%
\CSLLeftMargin{2. }%
\CSLRightInline{Atkinson, J.
\href{https://www.nasa.gov/feature/langley/what-is-earth-s-energy-budget-five-questions-with-a-guy-who-knows}{What
is earth's energy budget? Five questions with a guy who knows}. (2017).}

\leavevmode\vadjust pre{\hypertarget{ref-nclimgrid}{}}%
\CSLLeftMargin{3. }%
\CSLRightInline{Durre, I. \emph{et al.} NOAA nClimGrid-daily version 1
-- daily gridded temperature and precipitation for the contiguous united
states since 1951. (2022) doi:\url{https://doi.org/10.25921/c4gt-r169}.}

\leavevmode\vadjust pre{\hypertarget{ref-liebmann}{}}%
\CSLLeftMargin{4. }%
\CSLRightInline{Liebmann, B., Dole, R. M., Jones, C., Bladé, I. \&
Allured, D. \href{http://www.jstor.org/stable/26233052}{INFLUENCE OF
CHOICE OF TIME PERIOD ON GLOBAL SURFACE TEMPERATURE TREND ESTIMATES}.
\emph{Bulletin of the American Meteorological Society} \textbf{91},
1485--1492 (2010).}

\leavevmode\vadjust pre{\hypertarget{ref-cmip6}{}}%
\CSLLeftMargin{5. }%
\CSLRightInline{Eyring, V. \emph{et al.}
\href{https://doi.org/10.5194/gmd-9-1937-2016}{Overview of the coupled
model intercomparison project phase 6 (CMIP6) experimental design and
organization}. \emph{Geoscientific Model Development} \textbf{9},
1937--1958 (2016).}

\leavevmode\vadjust pre{\hypertarget{ref-usgcrp}{}}%
\CSLLeftMargin{6. }%
\CSLRightInline{USGCRP.
\href{https://www.globalchange.gov/browse/datasets/world-climate-research-programmes-wcrps-coupled-model-intercomparison-project-phase}{World
climate research programme's (WCRP's) coupled model intercomparison
project phase 3 (CMIP3) multi-model dataset}. \emph{Globalchange.gov}
(2010).}

\leavevmode\vadjust pre{\hypertarget{ref-ipcc}{}}%
\CSLLeftMargin{7. }%
\CSLRightInline{IPCC.
\href{https://www.ipcc.ch/report/ar6/wg1/downloads/report/IPCC_AR6_WGI_SPM.pdf}{Summary
for policymakers}. in \emph{Climate change 2021: The physical science
basis. Contribution of working group i to the sixth assessment report of
the intergovernmental panel on climate change} (eds. Masson-Delmotte, V.
et al.) (Cambridge University Press, 2021).}

\leavevmode\vadjust pre{\hypertarget{ref-mahony22}{}}%
\CSLLeftMargin{8. }%
\CSLRightInline{Mahony, C. R., Wang, T., Hamann, A. \& Cannon, A. J.
\href{https://doi.org/10.1002/joc.7566}{A global climate model ensemble
for downscaled monthly climate normals over north america}.
\emph{International Journal of Climatology} \textbf{42}, 5871--5891
(2022).}

\leavevmode\vadjust pre{\hypertarget{ref-ssp}{}}%
\CSLLeftMargin{9. }%
\CSLRightInline{Riahi, K. \emph{et al.}
\href{https://doi.org/10.1016/j.gloenvcha.2016.05.009}{The shared
socioeconomic pathways and their energy, land use, and greenhouse gas
emissions implications: An overview}. \emph{Global Environmental Change}
\textbf{42}, 153--168 (2017).}

\end{CSLReferences}



\end{document}
